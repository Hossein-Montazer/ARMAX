%%
%% IEEEtest.tex
%% (file is named IEEEtest_v12.tex for distribution purposes)
%% http://www.ctan.org/tex-archive/macros/latex/contrib/supported/IEEEtran/
%% Version 1.2 (7/27/2001) (IEEE release)
%% by Michael Shell
%% mshell@ece.gatech.edu
%%
%% This file demonstrates some problems found in the older IEEEtran.cls
%% files and serves as a test of usage. It should compile under older
%% as well as recent versions of IEEEtran.cls.
%%
%%**********************************************************************
%% Legal Notice:
%% This code is offered as-is without any warranty either
%% expressed or implied; without even the implied warranty of 
%% MERCHANTABILITY or FITNESS FOR A PARTICULAR PURPOSE!
%% User assumes all risk.
%% In no event shall IEEE or any contributor to this code
%% be liable for any damages or losses, including, but not limited to,
%% incidental, consequential, or any other damages, resulting from the
%% use or misuse of any information contained here.
%% 
%% All statements made here are the opinions of their respective
%% authors and are not necessarily endorsed by the IEEE.
%% 
%% This code is distributed under the Perl Artistic License 
%% ( http://language.perl.com/misc/Artistic.html ) 
%% and may be freely used, distributed and modified.
%% Please retain the contribution notices and credits.
%% 
%% Changes should be indicated by updating the version number in the
%% filename, i.e., "IEEEtest_v13.tex", as well as documented in the source.
%%  
%% Copyright 2000,2001 by Michael David Shell
%%**********************************************************************
%
%%**********************************************************************
%%
%%
%% V1.2 Changes (July 27, 2001):
%% 
%% 1. minor change in the backward compatibility code block
%%    to encompass the new optional argument in the biography
%%    environment of IEEEtran.cls version 1.5 and later.
%%    V1.1 used \CMPARstart where it should have used \PARstart.
%%
%%
%%*****
%%
%%
%% V1.1 Changes (March 15, 2001):
%% 
%% 1. Fixed incorrect comments that demonstrated how to revert
%%    back to Computer Modern fonts.
%%    
%% 2. Fixed a bug which would cause LaTeX compilation to hang
%%    if the user tried to define \PARstart in terms of
%%    \CMPARstart under the old (pre V1.3) IEEEtran.cls
%%    
%% 3. added \newif\ifcenterfigcaptions to compatibility code block
%%    (this flag is new to IEEEtran.cls V1.4)
%% 
%% Neither bug would affect the operation of an unmodified 
%% IEEEtest_v10.tex.
%% 
%% 
%%        *******   dvips note   *******  
%% 
%% Some LaTeX distributions have dvips configuration files (config.pdf)
%% which assume that the user will be working only with Computer Modern 
%% fonts. These configuration files remap certain character positions via 
%% the use of the dvips -G1 option (older versions of dvips may not have a
%% -Gx option). If this remapping is done to a document that contains other
%% fonts, such as Times, a problem will develop in the PostScript or PDF 
%% output. One symptom will be that the ligature "fi" will appear as the
%% British pound symbol.
%%    The dvips -G1 option should be used ONLY if the document is ENTIRELY 
%% in Computer Modern fonts. In all other cases, the -G0 option should be 
%% used.
%%    In particular, in MikTeX 2.0, you will have to use:
%% dvips -Ppdf -G0 <filename>
%% instead of:
%% dvips -Ppdf <filename>
%% if your document has any non Computer Modern fonts. This issue is not
%% specific to IEEEtran.cls and is applicable to all LaTeX documents.
%%             
%% Thanks to K. Sivakumar, A. Goreham, and K. Lee 
%% for identifying the problem and providing a solution.
%% 
%% 
%%*****
%%  
%% V1.0 was the initial release on January 30, 2001.
%%       
%%**********************************************************************  
%
%
% Compile with the old and then with the new IEEEtran.cls
% and see the difference.
% Leave the [10pt,twocolumn] options as they are - this
% test depends on controlled spacing.

\documentclass[10pt,twocolumn]{IEEEtran}
% to specify a filename, can use:
%\documentclass[10pt,twocolumn]{./IEEEtran_v15}

% I didn't like list-s
\hyphenation{lists}

% We need to obtain a command to tell us if a command has
% been previously defined so that we can determine which
% IEEEtran.cls is running. The \makeatletter stuff provides
% a way to "get at" the internal command \@ifundefined
% since it contains the @ character
\makeatletter
\def\ifundefined{\@ifundefined}
\makeatother

% V1.1 change: The font change commands have to be located
% in the preamble before the title or author is declared.
% If you uncomment these, you may also want to uncomment
% the redefinition of \PARstart around line 258 so that
% the big first letter is in the same font as the rest of
% the text.
% Here is how you can go back to the Computer Modern fonts
% if you wish:
%\renewcommand{\sfdefault}{cmss}
%\renewcommand{\rmdefault}{cmr}
%\renewcommand{\ttdefault}{cmtt}


\begin{document}


\title{A Test for IEEEtran.cls}

\author{Michael Shell\thanks{M. Shell is a Ph.D. student at the
School of Electrical and Computer Engineering
of the Georgia Institute of Technology,
Atlanta, Georgia, USA ~(email: mshell@ece.gatech.edu)}}

% The major version number of the class file will not
% be defined with the old IEEEtran.cls. So, we can use this fact
% to determine if we are running the old or the new class.
\ifundefined{IEEEtransversionmajor}{%
   % This block will be executed only if we are using the old
   % class. All we do is to make sure the V1.3 lengths and commands
   % actually exist so the code won't choke when it
   % doesn't find them.
   
   % This file doesn't need most of these definitions.
   % However, we'll provide them all in case somebody
   % wants to see what should be executed when compiling
   % a V1.3 or later IEEEtran.cls .tex file with a pre V1.3
   % IEEEtran.cls class file. In such a case, all you have to
   % do is copy this block to the start of your code. 
   % However, it won't fix any bugs in the old IEEEtran.cls!
   %
   % **** BACKWARD COMPATIBILITY CODE BLOCK START ****   
   \newlength{\IEEEilabelindent}
   \newlength{\IEEEilabelindentA}
   \newlength{\IEEEilabelindentB}
   \newlength{\IEEEelabelindent}
   \newlength{\IEEEdlabelindent}
   \newlength{\labelindent}
   \newlength{\IEEEiednormlabelsep}
   \newlength{\IEEEiedmathlabelsep}
   \newlength{\IEEEiedtopsep}

   \providecommand{\IEEElabelindentfactori}{1.0}
   \providecommand{\IEEElabelindentfactorii}{0.75}
   \providecommand{\IEEElabelindentfactoriii}{0.0}
   \providecommand{\IEEElabelindentfactoriv}{0.0}
   \providecommand{\IEEElabelindentfactorv}{0.0}
   \providecommand{\IEEElabelindentfactorvi}{0.0}
   \providecommand{\labelindentfactor}{1.0}
   
   \providecommand{\iedlistdecl}{\relax}
   \providecommand{\calcleftmargin}[1]{
                   \setlength{\leftmargin}{#1}
                   \addtolength{\leftmargin}{\labelwidth}
                   \addtolength{\leftmargin}{\labelsep}}
   \providecommand{\setlabelwidth}[1]{
                   \settowidth{\labelwidth}{#1}} 
   \providecommand{\usemathlabelsep}{\relax}
   \providecommand{\iedlabeljustifyl}{\relax}
   \providecommand{\iedlabeljustifyc}{\relax}
   \providecommand{\iedlabeljustifyr}{\relax}
 
   \newif\ifnocalcleftmargin
   \nocalcleftmarginfalse

   \newif\ifnolabelindentfactor
   \nolabelindentfactorfalse
   
   % in V1.4 of IEEEtran.cls
   \newif\ifcenterfigcaptions
   \centerfigcaptionsfalse
   
   % we need to provide the old IED environments
   % with a bogus optional argument
   \let\OLDitemize\itemize
   \let\OLDenumerate\enumerate
   \let\OLDdescription\description
   
   \renewcommand{\itemize}[1][\relax]{\OLDitemize}
   \renewcommand{\enumerate}[1][\relax]{\OLDenumerate}
   \renewcommand{\description}[1][\relax]{\OLDdescription}

   \providecommand{\pubid}[1]{\relax}
   \providecommand{\pubidadjcol}{\relax}
   \providecommand{\specialpapernotice}[1]{\relax}
   \providecommand{\overrideIEEEmargins}{\relax}
   
   % V1.1 change: use \let instead of \providecommand
   % This prevents LaTeX from hanging if the user ever
   % tried to redefine \PARstart in terms of \CMPARstart 
   \let\CMPARstart\PARstart 
   
   \let\OLDappendix\appendix
   \renewcommand{\appendix}[1][\relax]{\OLDappendix}
   
   \newif\ifuseRomanappendices
   \useRomanappendicestrue
   
   % V1.2 change: handle the optional biography environment argument
   % (the photo specifier) provided by IEEEtran V1.5 and later.
   % This is tricky because, under the new biography, the SECOND
   % argument is the non-optional one (the biography text).
   \let\OLDbiography\biography
   \let\OLDendbiography\endbiography
   \renewcommand{\biography}[2][\relax]{\OLDbiography{#2}}
   \renewcommand{\endbiography}{\OLDendbiography}
   % **** BACKWARD COMPATIBILITY CODE BLOCK END ****
   
   % alter the header to show we are using the older class
   \markboth{A Test for IEEEtran.cls--- {\tiny \bfseries
   [Running Older Class]}}{Shell: A Test for IEEEtran.cls}}{
   % END IF OLDER CLASS 
  
   % This block will be executed only if we are running 
   % the enhanced class
   % alter the header to show we are using the enhanced class
   \markboth{A Test for IEEEtran.cls--- {\tiny \bfseries
   [Running Enhanced Class
    V\IEEEtransversionmajor.\IEEEtransversionminor]}}%
   {Shell: A Test for IEEEtran.cls}}
% end of conditional


% Uncomment this line to render the big first letter in
% Computer Modern font. 
%\renewcommand{\PARstart}[2]{\CMPARstart{#1}{#2}}
% V1.1 change: This would hang if using the older IEEEtran.cls
% in IEEEtest V1.0, it is OK now.
%
%
% Here's how you would do invited papers:
%\specialpapernotice{(Invited Paper)}
% If you are binding copies of work generated with IEEEtran.cls,
% you may want to try:
%\overrideIEEEmargins
% These commands work OK here even though they are not in the preamble.
%(I wanted to put them after the backward compatibility code)


\maketitle
% here's how you get a publisher's ID mark with the new
% IEEEtran.cls.  If you want to use it, don't forget to
% also uncomment the \pubidadjcol command (which must be
% executed in the second text column) around line 434 below
%\pubid{0000--0000/00\$00.00~\copyright~2001 IEEE}

\begin{abstract}
This file is designed to demonstrate the problems with the
IEEEtran.cls class file which is currently
distributed (January 2001) by the IEEE. 
It is hoped that what you see here will convince you to 
switch to the version modified by Michael Shell and
Juergen von Hagen.
\end{abstract}

% no need for this single page document
% you may have to move \pubidadjcol (if used) if
% these are enabled
%\listoffigures
%\listoftables
%\tableofcontents

\section{Introduction}
% in V1.1, I made a mistake here by using \CMPARstart
\PARstart{H}{ere} is a little test for your IEEEtran class. IEEE
uses Times Roman font for \textit{Transactions} papers.

% Watch out to always have a blank line to end the paragraph
% after your \PARstart text as \PARstart can "swallow" a \begin{}.
% Comments don't count as blank lines!
\section{Itemized and Enumerated List Bugs}
Now a test of the itemized list (hrules added):
\vspace{0.4ex}\hrule
\begin{itemize}
\item The first characters of the text in every line of all the items
should line up nicely. This should be true even for long items
like this one that span more than one line. The lines should
{\bfseries not} be under the bullets!
\item Furthermore, the bullets should be slightly indented from the
main text above and below the list.
\item IEEE's demo file, ``\textit{Using the Document Class
IEEEtran.cls}" by Gerry Murray and Silvano Balemi, shows no
indention under the older IEEEtran.cls.
\item Yet, IEEE publications do use indention with itemized lists.
\end{itemize}
\hrule\vspace{0.4ex}
There should also be a very small padding space ($\approx$0.3ex)
above and below the list. However, this may be difficult to notice.
\par The same problems occur with enumerated lists.
\section{Math Font Woes}

A command like: \begin{verbatim}$A\mathbf{B}B$\end{verbatim}
should produce $A{\mathbf{B}}B$ with only the first ``B" being bold\hfill\\
Here is what your system produces: $A\mathbf{B}B$ \hfill\\
Is it correct for you? The last ``B" should not be bold.\hfill\\
The previous hack for this problem was to use the command:
\begin{verbatim}$A{\mathbf{B}}B$\end{verbatim}
which on your system yields: $A{\mathbf{B}}B$

% note that if you were to do something like this AFTER
% an appendix, then this should be a \section* command
\section{Reference Craziness-   discussion at end}

\begin{thebibliography}{11}
% Note the sample label "11" in thebibliography
% This is like the new IEEEtran class'
% \begin{enumerate}[\setlabelwidth{11}]
\bibitem{knuth}
D.~E. Knuth, {\em The Art of Computer Programming}.
\newblock Addison-Wesley, 1998.

\bibitem{nielsen2000}
T.~Nielsen {\em et al.}, ``3.28 Tb/s Transmission Over 3$\times$100 km of
  Nonzero-Dispersion Fiber Using Dual C- and L-Band Distributed Raman
  Amplification,'' {\em IEEE Photonics Technology Letters}, Vol.~12,
  pp.~1079--1081, Aug. 2000.

\bibitem{shell97}
M.~Shell, M.~D. Vaughn, L.~Dubertrand, D.~J. Blumenthal, P.-J. Rigole, and
  S.~Nilsson, ``Multinode Demonstration of a Multihop Wavelength-Routed
  All-Optical Packet-Switched Network,'' in {\em Optical Fiber Communications
  Conference}, Dallas, TX, pp.~91--92, Feb. 1997.

\bibitem{mccoy_phd}
K.~A. McCoy, {\em A Recirculating Optical Loop for Short-Term Data Storage}.
\newblock Ph.D.\ thesis, Georgia Institute of Technology, 1996.

\bibitem{bergstrom99}
P.~D. {Bergstrom Jr.}, M.~A. Ingram, A.~J. Vernon, J.~L.~A. Hughes, and
  P.~Tetali, ``A Markov Chain Model for an Optical Shared-Memory Packet
  Switch,'' {\em IEEE Transactions on Communications}, Vol.~47, pp.~1593--1603,
  Oct. 1999.

\bibitem{puckett}
D.~L. Puckett, and P.~Dibble {\em The Complete Rainbow Guide to OS-9
Level II, Volume I: A Beginners Guide to Windows}.
\newblock Falsoft Inc., 1987.

\bibitem{green96}
P.~E. {Green Jr.}, ``Optical Networking Update,'' {\em IEEE Journal on Selected
  Areas in Communications}, Vol.~14, pp.~764--779, June 1996.

\bibitem{kurzweil}
R.~Kurzweil, {\em The Age of Spiritual Machines}.
\newblock Viking, 1999.

\bibitem{kopka}
H.~Kopka and P.~W. Daly, {\em A Guide to \LaTeXe}.
\newblock Addison-Wesley, 1995.

\bibitem{williams}
W.~E. Williams, {\em Do the Right Thing}.
\newblock Hoover Institution Press, Stanford University, 1995.

\bibitem{lancaster}
D.~Lancaster, {\em Active Filter Cookbook}.
\newblock Synergetics, 1995.


\end{thebibliography}
\vspace{1ex}\hrule\vspace{3ex}

There may be several flaws in the references above:
% here we have to set the label width to our longest
% label, which happens to be "Second:"
% TIP: If you want to see what a variable's width is, you
% can use the TeX command \showthe\width-variable to 
% display it on the screen during compilation. Of course, 
% you can use \settowidth{\width-variable}{label-text}
% to set a width variable to the width of given text. 
% (This might be helpful to know when you need to find out
% which label is the widest) 
\begin{description}[\setlabelwidth{Second:}]
\item[First:] You may have gotten underfull hbox warnings
for bibliography entries one through nine.
\item[Second:] Note the alignment of the labels [10] and [11] with
respect to [9]. The ones digits should all be vertically aligned.
It is not correct to align by the ['s. It is the ]'s which should line
up.
% IEEE almost never plays games with item line spacing,
% but sometimes you will see double spaced items
% here's how you can skip a line between items:
% 1. You can put \setlength{\itemsep}{1.5ex} in the
%    list control argument above or 
% 2. You can put in bogus items   
%    \item \mbox{}
% 3. for more flexible control (not recommended for IEEE work):
%    \item \vspace*{-1.0ex}
%    negative heights subtract from the standard line height
%    positives add
\item[Lastly:] Watch reference number three. IEEE almost never breaks a
reference. The modified IEEEtran.cls discourages \LaTeX\ from doing this.
As a result, with the new version, some documents might produce an underfull
vbox warning which indicates that the spacing between two sections is a
bit larger than normal. However, this is typically the way it will be done in a real IEEE
publication.
\end{description}



% needed when using the publisher's ID mark
%\pubidadjcol

The description lists are where most IEEEtrans.cls have the biggest
problems. The item text should be block aligned and to the right of the
labels. Problems here make it a real nightmare to get nomenclature lists 
into decent form. The new IEEEtran.cls solves this problem and allows the 
user a way to specify the width of the longest label.
Labels are not necessarily \textit{italic} shape in IEEE publications.

\subsubsection{Note}{\ttfamily $\backslash$subsubsection} headings like this,
which one often sees in \textit{IEEE Transactions}, can falsely 
appear to be enumerated lists. One can always identify lists by the fact that 
the item text is block indented from the left. Also, sometimes the first 
word(s) of each list item are italicized.

\section{Conclusion}
% change our conclusion based on which class we run ;)
\ifundefined{IEEEtransversionmajor}{
I certainly hope that this draws some attention to the little 
problems of most IEEEtran.cls files today.}
{Tis better ain't it!? ;)}


\end{document}

% END of IEEEtest.tex ************


